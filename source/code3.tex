\begin{Verbatim}[commandchars=\\\{\},fontsize=\footnotesize]
\PY{n+nf}{condition}\PY{p}{(}\PY{l+s+s2}{\PYZdq{}x153\PYZdq{}}\PY{p}{)}\PY{p}{.}
\PY{n+nf}{condition\PYZus{}requirement}\PY{p}{(}\PY{l+s+s2}{\PYZdq{}x153\PYZdq{}}\PY{p}{,}\PY{l+s+s2}{\PYZdq{}node\PYZdq{}}\PY{p}{,}\PY{l+s+s2}{\PYZdq{}example\PYZdq{}}\PY{p}{)}\PY{p}{.}
\PY{n+nf}{condition\PYZus{}requirement}\PY{p}{(}\PY{l+s+s2}{\PYZdq{}x153\PYZdq{}}\PY{p}{,}\PY{l+s+s2}{\PYZdq{}variant\PYZdq{}}\PY{p}{,}\PY{l+s+s2}{\PYZdq{}example\PYZdq{}}\PY{p}{,}\PY{l+s+s2}{\PYZdq{}bzip\PYZdq{}}\PY{p}{,}\PY{l+s+s2}{\PYZdq{}True\PYZdq{}}\PY{p}{)}\PY{p}{.}
\PY{n+nf}{imposed\PYZus{}constraint}\PY{p}{(}\PY{l+s+s2}{\PYZdq{}x153\PYZdq{}}\PY{p}{,}\PY{l+s+s2}{\PYZdq{}depends\PYZus{}on\PYZdq{}}\PY{p}{,}\PY{l+s+s2}{\PYZdq{}example\PYZdq{}}\PY{p}{,}\PY{l+s+s2}{\PYZdq{}bzip2\PYZdq{}}\PY{p}{)}
\end{Verbatim}
