\section{Splicing}
\begin{figure}
  \begin{tikzpicture}
    [> = stealth,
    every node/.append style = {
      inner sep = 1pt
    },
    tnode/.style = {
      shape = rectangle,
      draw = black,
      minimum width = \minwidthoval
    },
    hnode/.style = {
      shape = rounded rectangle,
      draw = black,
      minimum width = \minwidthoval  
    }
    ]
    % Main box
    \fill[lightgray, opacity=0.5] (-0.6,-0.5) rectangle (5.3, 2.5);
    \node[anchor=south east]  at (5, 2) {Already built specs};

    % T ^H' box
    \fill[cyan, opacity=0.1] (-0.6, -4) rectangle (2.3, -1);
    \node[anchor=south east] at (2.05, -1.4) {\spec{T \textasciicircum{}H'}};
    

    % T ^H' ^Z box
    \fill[purple, opacity=0.1] (2.4, -4) rectangle (5.3, -1);
    \node[anchor=south east] at (5, -1.4) {\spec{T\textasciicircum{}H'\textasciicircum{}Z@1.0}};
    
    \node[tnode] (t) at (0,2) {$T$};
    \node[tnode] (h) at (1,1) {$H$};
    \node[tnode] (z10) at (0,0) {$Z@1.0$};
    \node[hnode] (h') at (4, 1) {$H'$};
    \node[hnode] (z11) at (3, 0) {$Z@1.1$};
    \node[hnode] (s) at (5, 0) {$S$};
    \node[tnode] (splicet) at (0, -1.5) {$T$};
    \node[hnode] (spliceh') at (1, -2.5) {$H'$};
    \node[hnode] (splicez) at (0, -3.5) {$Z@1.1$};
    \node[hnode] (splices) at (2, -3.5) {$S$};
    \node[tnode] (insplicet) at (3, -1.5) {$T$};
    \node[hnode] (inspliceh') at (4, -2.5) {$H'$};
    \node[tnode] (insplicez) at (3, -3.5) {$Z@1.0$};
    \node[hnode] (insplices) at (5, -3.5) {$S$};
    \path[->] (t) edge (h);
    \path[->] (t) edge (z10);
    \path[->] (h) edge (z10);
    \path[->] (h') edge (z11);
    \path[->] (h') edge (s);
    \path[->] (splicet) edge (spliceh');
    \path[->] (splicet) edge (splicez);
    \path[->] (spliceh') edge (splicez);
    \path[->] (spliceh') edge (splices);
    \path[->] (insplicet) edge (inspliceh');
    \path[->] (insplicet) edge (insplicez);
    \path[->] (inspliceh') edge (insplicez);
    \path[->] (inspliceh') edge (insplices);
    \draw[dashed, bend left = 55, ->] (splicet) to (t);
    \draw[dashed, bend left = 25, ->] (insplicet) to (t);
    \draw[dashed, ->] (inspliceh') to (h');
  \end{tikzpicture}
  \caption{An example of how splicing combines spec-DAGs while maintaining build
    provenance. The specs in gray, \spec{T \textasciicircum{}H \textasciicircum{}Z@1.0} (rectangular nodes) and
    \spec{H' \textasciicircum{}S \textasciicircum{}Z@1.1} (rounded nodes), are already built. A spec
    conforming to \spec{T \textasciicircum{}H'} can be satisfied with a
    splice of the already built \spec{H'} into already built \spec{T} (blue
    background). A spec conforming to \spec{T \textasciicircum{}H' \textasciicircum{}Z@1.0} can be satisfied
    by splicing \spec{Z@1.0} back into the previous spec (red background). The
  build provenance for spliced specs are denoted with dashed lines.}
\label{fig:splice-example}
\Description{A graphical representation of transitive and intransitive splices.}
\end{figure}

Supporting ABI compatibility presents two key package management opportunities.
The first is to avoid the unnecessary cascade of rebuilds for transitive
dependent packages when swapping ABI-compatible dependencies. Similarly,
deploying a package binary built with a reference library (like \package{MPICH})
onto systems with custom ABI-compatible implementations should not trigger rebuilds.

As discussed in \cref{sec:background-pm}, updating a dependency in a source
build can cause a series of package rebuilds for everything that directly
and transitively depends on it even when the updated package is ABI-compatible
with the version of the package that it is updating.  Instead, only the updated
package needs to be rebuilt and the RPATHs of all dependents updated to the
new binary.

The deployment scenario is even simpler.
Recall from \cref{sec:introduction} the problem of deploying a package on a
cluster using \package{Cray MPICH} when the package binary was built against
a general implementation {MPICH} on a build server.
Notably, \package{Cray MPICH} is usually a binary that
exists only on the cluster.
Assuming the general implementation conforms to the \package{Cray MPICH} ABI,
deployment of the package binary should only require that it point to \package{Cray MPICH}.

These scenarios introduce two challenges for Spack:
\begin{enumerate}
\item How should Spack represent binaries built against one
  implementation of an interface and deployed against another?
  %This
  %representation should make it clear how to relocate each binary and provide
  %full build provenance for reproducibility.
\item How should developers specify ABI-compatibility information to Spack such
  that these solutions are automated?
\end{enumerate}
In this section we present \textit{splicing} as a solution to the first
challenge and focus on its mechanics and representation. We also present a
solution to the second challenge of automating splicing in
\cref{sec:implementation}.  Splicing is a model of describing package
installations built against one shared library and deployed against
another, while maintaining full build provenance.

\subsection{The Mechanics of Splicing}
\label{sec:splice-mechanics}
We present the mechanics of splicing through a stripped down synthetic scenario,
as depicted in \cref{fig:splice-example}: There are two pre-compiled packages,
\spec{T} and \spec{H'}, conforming to \spec{T \textasciicircum{}H \textasciicircum{}Z@1.0} (rectangular nodes)
and \spec{H' \textasciicircum{}S \textasciicircum{}Z@1.1} (rounded nodes). The user is requesting a package
installation conforming to \spec{T \textasciicircum{}H'}, and Spack knows that \spec{H} and
\spec{Z@1.0} are ABI-compatible with \spec{H'} and \spec{Z@1.1},
respectively. Currently, the best Spack could do to satisfy the request would be
to recompile \spec{T} against \spec{H'}; however, our goal is to satisfy the
user's request without recompiling.

At the level of spec DAGs, the solution is as simple as replacing the \spec{T}
dependency on \spec{H} with the full spec for \spec{H'}, which we call splicing;
however, there are a few subtleties to acknowledge. The first is the
handling of the shared \spec{Z} dependency. Simply copying in the full spec for
\spec{H'} would leave two copies of the \spec{Z} dependency at different
versions.  Since there can only be one version of any dependency in the link-run
graph for a spec, splicing must have a way to ``break ties.'' In general, we
consider splices to be \emph{transitive}, namely that splicing a dependent spec
also replaces all shared dependencies between the root and the spec being
spliced in.

If the user instead requested \spec{T \textasciicircum{}H' \textasciicircum{}Z@1.0} (\ie{}
explicitly constraining \spec{Z}), this can be solved with another
splice. Splicing \spec{Z} back into the the initial transitively spliced spec
(blue background in \cref{fig:splice-example}), would produce an
\emph{intransitive} splice (red background in \cref{fig:splice-example}), where
\spec{Z@1.1} dependency for \spec{H'} is replaced with \spec{Z@1.0}, and the
dependency for \spec{T} is restored.

The second subtlety is related to the \emph{build provenance} of spliced
specs. Note that \spec{T \textasciicircum{}H' \textasciicircum{}Z@1.1} \emph{could} have been how the related
binaries were built, and thus a naive implementation attempting to reproduce
this spec would build it directly. Yet, this is not actually how the binaries
corresponding to this spec were built. First, \spec{T \textasciicircum{}H \textasciicircum{}Z@1.0} and
\spec{H' \textasciicircum{}Z@1.1} were built, and then they were spliced together. In order to
maintain reproducibility, Spack needs to distinguish spliced specs
from non-spliced specs and should be able reproduce the whole process of how a
spliced spec was constructed by building the initial specs and then splicing
them together, rather than just building the resulting spec. Therefore, we
augment spliced specs with a reference to the spec from which they are derived
that we call the \emph{build spec} (denoted by the dashed line in \cref{fig:splice-example}).

The final subtlety is the handling of build dependencies (not
pictured). Splicing is only a consideration for concrete specs which have
already been built (\ie{} there is an existing binary to be
relinked). On the other hand, recall that build dependencies need not be present
\emph{after} being used to build the spec's binary.
Therefore, build dependencies are removed from the spec whose dependency
has been spliced, since they are no longer relevant to the runtime
representation. By keeping around the aforementioned build spec, removing these
build dependencies from the spliced spec does not sacrifice reproducibility.

\subsection{Patching Spliced Binaries}
\label{sec:splice-relocation}
Recall from \cref{sec:spack-relocation} that all Spack requires for binary
relocation is a mapping from the current installation paths of shared libraries
to the desired relocation paths of those libraries. This machinery can easily be
reused for patching binaries from spliced specs by generalizing
\emph{relocation} to \emph{rewiring}. Instead of changing the paths of the same
shared library to different locations, the shared library of the old node in the
spec before it was spliced is patched to be the new shared library of the
analogous node after it was spliced. This rewiring demonstrates the need for the
build spec in a spliced spec, since it allows for Spack to determine the mapping
between how the binary was initially installed to how it needs to be patched in
order to conform to the spliced spec. Build specs also make it simple to
determine whether a spec has been spliced at all, since only spliced specs will
have an associated build spec.
