\section{Spack}
\label{sec:spack}

Spack is a source {\it and} binary package manager, widely used in HPC. This
section describes Spack's \textit{packages}, \textit{specs},
\textit{concretizer}, and \textit{binary relocation}, core components of
configuration and dependency resolution that provide needed context for
splicing.

\subsection{Specs}
\label{sec:spack-specs}
\textit{Specs} are the core of Spack; they allow users to concisely specify
build configuartions and constraints. Each spec has six attributes:
\begin{enumerate*}
  \item the package name;
  \item the version to build;
  \item variant values (\ie{} compile-time options);
  \item the target operating system;
  \item the target microarchitecture;
  \item and the specs of the packages it depends on (and their attributes).
\end{enumerate*}
%
\begin{table}
  \begin{tabularx}{1.0\columnwidth}{|c|X|X|}
    \hline
    Spec Sigil & Example & Meaning \\
    \hline
    \spec{@} & \spec{hdf5@1.14.5} & Require version \\
    \hline
    \spec{+} & \spec{hdf5+cxx} & Require variant \\
    \spec{\textasciitilde} & \spec{hdf5\textasciitilde{}mpi} & Disable variant \\
    \hline
    \spec{\textasciicircum} & \spec{hdf5 \textasciicircum{}zlib} & Depends on (link-run)\\
    \spec{\%} & \spec{hdf5\%clang} & Depends on (build)\\
    \hline
    \spec{key=value} & \spec{hdf5 target=icelake} \newline
                       \spec{hdf5 api=default}  & Require variant or target
                                                  to hold value \\
    \hline
  \end{tabularx}
  \caption{Examples of spec syntax, and their meaning.}
  \Description{A table detailing the spec CLI syntax used by Spack.}
  \label{fig:spec-syntax}
\end{table}

%
Spack provides a concise syntax for specs, and \Cref{fig:spec-syntax}
shows a subset of simple examples needed to understand this paper.
A user typically provides an incomplete description,
or \textit{abstract spec}, by not explicitly constraining all
attributes. In contrast, a spec with all attributes set is called a
\textit{concrete spec}. Concrete specs can be built and installed, as they
contain all information the build environment might need to query in order
to configure and build.
The rest of the paper is concerned mainly with concrete specs.

Specs are recursive in that dependencies (attribute (6) above) are specs
themselves. They are therefore represented in memory as directed acyclic
multigraphs. Nodes correspond to packages with their attributes, a directed
edge $(A, B)$ denoting that package $A$ \textit{depends on} package $B$, and two
edge sets corresponding to build, and link-run dependencies.  Taking the union
of all edge sets and forming a single DAG admits a performant hashing scheme on
specs, allowing for efficient reasoning about equality between specs.

\subsection{Packages}
\label{sec:spack-packages}
A Spack \textit{package} defines the build process for a software
product using available options. It defines a combinatorial space of build
options that can include software versions, optional features,
compile-time flags, dependencies on other packages, potential conflicts with
other packages, and environmental constraints (\eg{} requiring an x86\_64
system, requiring a CUDA installation, \etc{}).
The software is described in a DSL embedded in Python that uses a Python
class to represent every build configuration, a \textit{parameterized} build
process, and the artifacts (\eg{} tarballs and patches) needed to build.
%
\begin{figure}
\begin{Verbatim}[commandchars=\\\{\},numbers=left,xleftmargin=5mm,fontsize=\footnotesize]
\PY{k}{class} \PY{n+nc}{Example}\PY{p}{(}\PY{n}{Package}\PY{p}{)}\PY{p}{:}
\PY{+w}{    }\PY{l+s+sd}{\PYZdq{}\PYZdq{}\PYZdq{}Example depends on zlib, mpi and optionally bzip2\PYZdq{}\PYZdq{}\PYZdq{}}
    \PY{c+c1}{\PYZsh{} This package provides two versions}
    \PY{n}{version}\PY{p}{(}\PY{l+s+s2}{\PYZdq{}}\PY{l+s+s2}{1.1.0}\PY{l+s+s2}{\PYZdq{}}\PY{p}{)}
    \PY{n}{version}\PY{p}{(}\PY{l+s+s2}{\PYZdq{}}\PY{l+s+s2}{1.0.0}\PY{l+s+s2}{\PYZdq{}}\PY{p}{)}

    \PY{c+c1}{\PYZsh{} Optional bzip support}
    \PY{n}{variant}\PY{p}{(}\PY{l+s+s2}{\PYZdq{}}\PY{l+s+s2}{bzip}\PY{l+s+s2}{\PYZdq{}}\PY{p}{,} \PY{n}{default}\PY{o}{=}\PY{k+kc}{True}\PY{p}{)}

    \PY{c+c1}{\PYZsh{} Depends on bzip2 when bzip support is enabled}
    \PY{n}{depends\PYZus{}on}\PY{p}{(}\PY{l+s+s2}{\PYZdq{}}\PY{l+s+s2}{bzip2}\PY{l+s+s2}{\PYZdq{}}\PY{p}{,} \PY{n}{when}\PY{o}{=}\PY{l+s+s2}{\PYZdq{}}\PY{l+s+s2}{+bzip}\PY{l+s+s2}{\PYZdq{}}\PY{p}{)}
    \PY{c+c1}{\PYZsh{} Version 1.0.0 depends on an older version of zlib}
    \PY{n}{depends\PYZus{}on}\PY{p}{(}\PY{l+s+s2}{\PYZdq{}}\PY{l+s+s2}{zlib@1.2}\PY{l+s+s2}{\PYZdq{}} \PY{n}{when}\PY{o}{=}\PY{l+s+s2}{\PYZdq{}}\PY{l+s+s2}{@1.0.0}\PY{l+s+s2}{\PYZdq{}}\PY{p}{)}
    \PY{c+c1}{\PYZsh{} Version 1.1.0 depends on a newer version of zlib}
    \PY{n}{depends\PYZus{}on}\PY{p}{(}\PY{l+s+s2}{\PYZdq{}}\PY{l+s+s2}{zlib@1.3}\PY{l+s+s2}{\PYZdq{}}\PY{p}{,} \PY{n}{when}\PY{o}{=}\PY{l+s+s2}{\PYZdq{}}\PY{l+s+s2}{@1.1.0}\PY{l+s+s2}{\PYZdq{}}\PY{p}{)}
    \PY{c+c1}{\PYZsh{} Depends on some implementation of MPI}
    \PY{n}{depends\PYZus{}on}\PY{p}{(}\PY{l+s+s2}{\PYZdq{}}\PY{l+s+s2}{mpi}\PY{l+s+s2}{\PYZdq{}}\PY{p}{)}

    \PY{c+c1}{\PYZsh{}\PYZsh{}\PYZsh{} }
    \PY{c+c1}{\PYZsh{} example@1.1.0 can be spliced in for example@1.0.0}
    \PY{n}{can\PYZus{}splice}\PY{p}{(}\PY{l+s+s2}{\PYZdq{}}\PY{l+s+s2}{example@1.0.0}\PY{l+s+s2}{\PYZdq{}}\PY{p}{,} \PY{n}{when}\PY{o}{=}\PY{l+s+s2}{\PYZdq{}}\PY{l+s+s2}{@1.1.0}\PY{l+s+s2}{\PYZdq{}}\PY{p}{)}
    \PY{c+c1}{\PYZsh{} example@1.1.0+bzip can be spliced in for}
    \PY{c+c1}{\PYZsh{} example\PYZhy{}ng@2.3.2+compat}
    \PY{n}{can\PYZus{}splice}\PY{p}{(}\PY{l+s+s2}{\PYZdq{}}\PY{l+s+s2}{example\PYZhy{}ng@2.3.2+compat}\PY{l+s+s2}{\PYZdq{}}\PY{p}{,} \PY{n}{when}\PY{o}{=}\PY{l+s+s2}{\PYZdq{}}\PY{l+s+s2}{@1.1.0+bzip}\PY{l+s+s2}{\PYZdq{}}\PY{p}{)}
\end{Verbatim}

\caption{An example \texttt{package.py} file demonstrating constraints in
    Spack's embedded DSL. This package also contains our new augmentation to the
    packaging language, \directive{can\_splice}, which we introduce in
    \cref{sec:impl-new-directive}.}
  \label{fig:example-pkg}
  \Description{A snippet of Python code demonstrating the Spack packaging DSL.}
\end{figure}

%
\Cref{fig:example-pkg} shows a simple Spack package. Package
configuration features, such as \directive{version}, \directive{depends\_on},
and \directive{variant}, are specified with \textit{directives}. Directives
serve to define the configuration space of a package and may introduce
constraints on the package or its dependencies.\footnote{Some directives
  simply attach metadata, such as the Github usernames of the maintainers,
  licensing information, or websites associated with the package.}
In contrast to most package managers, where a package description specifies
a single software configuration, Spack packages are {\it conditional}, and most
directives accept a {\tt when} argument. For example, the \directive{depends\_on}
directive on line 13 constrains the {\tt zlib} dependency to version {\tt 1.2}
{\it when} the example package is at version {\tt 1.0.0}. Similarly, the
directive on line 15 says that
{\tt example} version {\tt 1.1.0} requires a newer {\tt zlib} version: {\tt 1.3}.
Constraints in package directives are specified using the spec syntax
(\Cref{sec:spack-specs}). Directives below \texttt{\#\#\#}
are part of our DSL extension (\cref{sec:impl-new-directive}).

For our purposes, there are two major classes of dependencies in Spack: build
and link-run. Build dependencies are packages that a node needs to
execute its build process; these can be compilers (\eg{} \package{GCC}, or
\package{clang}), (meta-)build systems(\eg{} \package{autotools},
\package{CMake}, or \package{Ninja}), or even interpreters for build-time glue
code (\eg{} \package{Python} or \package{Perl}). Link-run dependencies are the
packages that a node needs either at compile-time for linking (\ie shared
object files for dynamic linking and assembly code for static linking), or at
runtime to use either as a subprocess or as a dynamically loaded library.
%
\subsection{The Concretizer}
\label{sec:spack-concretizer}
Spack's dependency resolver, or {\it concretizer}, takes abstract specs
requested by a user and produces concrete specs that are valid according
to constraints in package files. Since specs are recursive, this also
includes resolving concrete specs for all of a package's
dependencies. Consider the following abstract spec for the package from
\cref{fig:example-pkg}, \spec{example@1.0.0}; one such concretization could be:

\begin{footnotesize}
\par\vspace{0.8em}\noindent\colorbox{whitesmoke}{\begin{minipage}{0.98\columnwidth}\begin{Verbatim}[commandchars=\\\{\},fontsize=\footnotesize]
example@1.0.0 +bzip arch=linux\PYZhy{}centos8\PYZhy{}skylake
    \PYZca{}bzip2@1.0.8 \PYZti{}debug+pic+shared arch=linux\PYZhy{}centos8\PYZhy{}skylake
    \PYZca{}zlib@1.2.11 +optimize+pic+shared arch=linux\PYZhy{}centos8\PYZhy{}skylake
    \PYZca{}mpich@3.1 pmi=pmix arch=linux\PYZhy{}centos8\PYZhy{}skylake
\end{Verbatim}
\end{minipage}}\vspace{0.8em}
\end{footnotesize}

Dependency resolution when considering only compatible versions is known to be
NP-complete~\cite{dicosmo2006edos, cox2016version}, and Spack must also decide
compatible build options, operating systems, and micro-architectures.  Spack
attempts to maximize its reuse of already-built components, either
previously installed locally or present in some remote cache, which we
collectively call \textit{reusable specs}. To ensure completeness and optimality
(\ie{} always finding the optimal solution if one exists) while maintaining
tractability, Spack implements its concretizer using the Answer Set Programming
(ASP) system Clingo~\cite{gebser2016potassco, gamblin2022asp}. ASP is a logic
programming paradigm, similar to Datalog, where problems are specified as
first-order logic programs (\ie{} facts and deductive rules with variables)
extended with a non-deterministic choice construct. ASP is distinguished from
other logic programming paradigms by its lack of an operational semantics;
instead, ASP programs are first \textit{grounded} into propositional logic
programs without variables, and then solved for their \textit{stable models}
using techniques from the SAT/SMT solving community such as the classic
Davis–Putnam–Logemann–Loveland (DPLL) algorithm along with modern extensions
like Conflict-Driven Clause Learning (CDCL)~\cite{davis1960computing,
  davis1962machine, moskewicz2001chaff}. Since a given program may have many
models in the presence of non-deterministic choices, Clingo allows
optimization objectives to be defined over models of a program.

Spack's concretizer is implemented in three stages. First, constraints from the
package classes, reusable specs, and the user-provided abstract specs are
compiled to an encoding of facts in ASP. Then, these facts are included in a
logic program describing the formal constraints of spec concretization (\eg{}
every spec must be resolved to exactly one version, every variant must have a
chosen value, \etc{}) and optimization objectives (\eg{} maximize the reusable
specs, use the newest satisfying version of every package, \etc{}), which is
then run with Clingo to solve for stable models. Finally, the optimal model
is processed by an interpreter to construct concrete specs for further use by
Spack. Notably, this output includes which specs are reused (\ie{} those which have
already been built), and which ones must be built from source.
%
Further details on the concretizer implementation, its optimization criteria,
and the implementation of package reuse can be found in prior work~\cite{gamblin2022asp}.

%
\subsection{Binary Relocation}
\label{sec:spack-relocation}
Once Spack builds a package, it needs to ensure that each executable and library knows
where to find its dependencies. On systems where there is only one version of each
library, managed by the system package manager, the search path for
libraries is usually stored either in global system configuration (\eg{} {\tt ldconfig} or {\tt
  ld.so.conf}), or in an environment variable (\eg{} \texttt{LD\_LIBRARY\_PATH}). A
naive install algorithm would simply append installation paths for Spack-built binaries
to system search paths. This could lead to unintended crashes in both system
libraries and Spack-built libraries, since each could attempt to access unexpected,
incompatible libraries.
%
Taking inspiration from the Nix package
manager~\cite{dolstra2004nix, dolstra2008nixos},
Spack instead uses RPATHs~\cite{gamblin2015spack}.
RPATHs allow for embedding the location of shared libraries directly into a
binary without environment modification. All
Spack packages are installed in a user-defined prefix and all dependencies are
embedded as RPATHs.

Unlike Nix, Spack supports {\it binary relocation}. Spack build caches can be installed
in any location, including user home directories. The deployment locations of a binary
package and its dependencies may {\it not} be the same at runtime as at build time. To
ensure correct execution, Spack rewrites references to installation locations in
installed files. It creates a mapping from binaries' original
install locations to new locations, and it patches all occurrences of the old locations.
%
For simple cases, where the install location is shorter than the build location, Spack
uses simple patching logic. For more complex installs, it can use the \texttt{patchelf}
tool to lengthen paths~\cite{dolstra-patchelf}.
%
Spack can build binaries on a node's local filesystem,
relocate them to a network file system, push them to a remote server for hosting
reusable binaries, and install them again on a separate cluster.

