\section{Automatic Splicing}
\label{sec:implementation}
In this section we describe the implementation of automatic splicing in Spack.
The concretizer synthesizes spliced solutions from pre-installed specs when
package developer-specified declarative constraints on splice validity are
satisfied. As described in previous sections, this allows specs from a
buildcache to be built using general implementations of common interfaces, like
\package{MPICH}, and deployed without user intervention against specialized
implementations, such as \package{CrayMPICH}.  It also enables updating
dependencies in a way that does not require \emph{rebuilding the world}.

Several changes needed to be made to Spack. A new directive was added
to the packaging language for expressing
ABI-compatibility (\cref{sec:impl-new-directive}). The ASP encoding of
pre-installed specs needed to be changed to allow for splicing (\cref{sec:impl-new-encoding}).
Finally, logic was added to the concretizer to synthesize and execute splices
(\cref{sec:impl-splice-concretizer}). \Cref{sec:impl-basic-asp} provides an
introduction to the ASP encoding and techniques used in Spack's
concretizer.\footnote{Most ASP examples are simplified to the core logic
  necessary to understand the approach in this work. Spack's full concretizer
  has a number of indirections and augmentations to support a diverse array of
  features that are orthogonal to splicing and binary reuse.}

\subsection{ASP in the concretizer}
\label{sec:impl-basic-asp}
Recall from \cref{sec:spack-concretizer} that the ASP portion of Spack's
concretizer takes four inputs: compiled ASP facts of the user's requested spec,
compiled ASP facts from all package definition classes, an encoding of the
concrete specs that it can reuse, and a logic program implementing
Spack's concretization semantics. Specs are encoded through two relations:
\asp{node} and \asp{attr}. The unary \asp{node} relation simply takes the name
of the package that it is encoding to indicate its presence. The higher-order,
variable arity \asp{attr} relation describes how various attributes are
attached to nodes. The first argument of \asp{attr}
is always the name of the attribute, the second is the node to which it applies,
and all other arguments depend on the particular attribute. By using a
higher-order \asp{attr} relation, rather than individual relations for each
potential attribute, common patterns that relate to all of the potential
attributes can be reused (such as \asp{impose} later in this section). The
following is the encoding of \spec{example+bzip \textasciicircum{}bzip2}:
\par\vspace{0.8em}\noindent\colorbox{whitesmoke}{\begin{minipage}{0.98\columnwidth}\begin{Verbatim}[commandchars=\\\{\},fontsize=\footnotesize]
\PY{n+nf}{node}\PY{p}{(}\PY{l+s+s2}{\PYZdq{}example\PYZdq{}}\PY{p}{)}\PY{p}{.}
\PY{n+nf}{node}\PY{p}{(}\PY{l+s+s2}{\PYZdq{}bzip2\PYZdq{}}\PY{p}{)}\PY{p}{.}
\PY{n+nf}{attr}\PY{p}{(}\PY{l+s+s2}{\PYZdq{}depends\PYZus{}on\PYZdq{}}\PY{p}{,}\PY{n+nf}{node}\PY{p}{(}\PY{l+s+s2}{\PYZdq{}example\PYZdq{}}\PY{p}{)}\PY{p}{,}\PY{n+nf}{node}\PY{p}{(}\PY{l+s+s2}{\PYZdq{}bzip2\PYZdq{}}\PY{p}{)}\PY{p}{,}\PY{l+s+s2}{\PYZdq{}link\PYZhy{}run\PYZdq{}}\PY{p}{)}\PY{p}{.}
\PY{n+nf}{attr}\PY{p}{(}\PY{l+s+s2}{\PYZdq{}variant\PYZdq{}}\PY{p}{,}\PY{n+nf}{node}\PY{p}{(}\PY{l+s+s2}{\PYZdq{}example\PYZdq{}}\PY{p}{)}\PY{p}{,}\PY{l+s+s2}{\PYZdq{}bzip\PYZdq{}}\PY{p}{,}\PY{l+s+s2}{\PYZdq{}True\PYZdq{}}\PY{p}{)}\PY{p}{.}
\end{Verbatim}
\end{minipage}}\vspace{0.8em}
Notably, the spec encoded above is not yet concrete, since not all attributes
have been set (\eg{} the versions of the packages). The logic program
implementing the concretizer describes how to take user provided abstract
specs and the compiled package definitions to set \emph{all} attributes on the
nodes. Attributes range not only over the six for concrete specs
described in \cref{sec:spack-specs}, but also can describe various integrity
properties of the whole spec DAG as it relates to a particular node.
There are over twenty different attributes considered in the concretizer in all.

The directives from package definitions are compiled to analogous ASP facts
describing constraints on the solution space. The following is an example of the
facts related to versions for the package from \cref{fig:example-pkg}, declaring
two unconditional versions for the package \spec{1.1.0} and \spec{1.0.0}:
%
\par\vspace{0.8em}\noindent\colorbox{whitesmoke}{\begin{minipage}{0.98\columnwidth}\begin{Verbatim}[commandchars=\\\{\},fontsize=\footnotesize]
\PY{n+nf}{pkg\PYZus{}fact}\PY{p}{(}\PY{l+s+s2}{\PYZdq{}example\PYZdq{}}\PY{p}{,}\PY{n+nf}{version\PYZus{}declared}\PY{p}{(}\PY{l+s+s2}{\PYZdq{}1.1.0\PYZdq{}}\PY{p}{)}\PY{p}{)}\PY{p}{.}
\PY{n+nf}{pkg\PYZus{}fact}\PY{p}{(}\PY{l+s+s2}{\PYZdq{}example\PYZdq{}}\PY{p}{,}\PY{n+nf}{version\PYZus{}declared}\PY{p}{(}\PY{l+s+s2}{\PYZdq{}1.0.0\PYZdq{}}\PY{p}{)}\PY{p}{)}\PY{p}{.}
\end{Verbatim}
\end{minipage}}\vspace{0.8em}
The logic program then has the related rule for selecting versions which can
be read as ``choose exactly one version for each node from the
declared versions.'':
\par\vspace{0.8em}\noindent\colorbox{whitesmoke}{\begin{minipage}{0.98\columnwidth}\begin{Verbatim}[commandchars=\\\{\},fontsize=\footnotesize]
\PY{l+m+mi}{1} \PY{p}{\PYZob{}} \PY{n+nf}{attr}\PY{p}{(}\PY{l+s+s2}{\PYZdq{}version\PYZdq{}}\PY{p}{,}\PY{n+nf}{node}\PY{p}{(}\PY{n+nv}{Name}\PY{p}{)}\PY{p}{,}\PY{n+nv}{Ver}\PY{p}{)}
    \PY{l+s+sAtom}{:} \PY{n+nf}{pkg\PYZus{}fact}\PY{p}{(}\PY{n+nv}{Name}\PY{p}{,} \PY{n+nf}{version\PYZus{}declared}\PY{p}{(}\PY{n+nv}{Ver}\PY{p}{)}
 \PY{p}{\PYZcb{}} \PY{l+m+mi}{1} \PY{p}{:\PYZhy{}}  
  \PY{n+nf}{attr}\PY{p}{(}\PY{l+s+s2}{\PYZdq{}node\PYZdq{}}\PY{p}{,}\PY{n+nf}{node}\PY{p}{(}\PY{n+nv}{Name}\PY{p}{)}\PY{p}{)}\PY{p}{.}
\end{Verbatim}
\end{minipage}}\vspace{0.8em}
There are analogous corresponding package facts and concretizer rules for each
associated attribute for concrete spec (\eg{} variant values, dependencies, \etc{}).

\subsubsection{Handling conditional constraints}
\label{sec:concretizer-imposition}
Many package directives have a \texttt{when} argument for expressing
conditional constraints. These require a little more machinery to express in
ASP. Each condition is given a unique identifier, the requirements
for the condition to be satisfied, and the constraint imposed on the resulting
models if those conditions hold. Below is the compiled representation of the
conditional dependency from line 11 of the package in \cref{fig:example-pkg}. It
declares a condition with unique identifier \asp{"x153"}, with conditions that
there is a node \asp{"example"} with variant \asp{"bzip"} set
to \asp{"True"}. The condition then imposes the constraint that the \asp{"example"}
node also depends on \asp{"bzip2"}.
\par\vspace{0.8em}\noindent\colorbox{whitesmoke}{\begin{minipage}{0.98\columnwidth}\begin{Verbatim}[commandchars=\\\{\},fontsize=\footnotesize]
\PY{n+nf}{condition}\PY{p}{(}\PY{l+s+s2}{\PYZdq{}x153\PYZdq{}}\PY{p}{)}\PY{p}{.}
\PY{n+nf}{condition\PYZus{}requirement}\PY{p}{(}\PY{l+s+s2}{\PYZdq{}x153\PYZdq{}}\PY{p}{,}\PY{l+s+s2}{\PYZdq{}node\PYZdq{}}\PY{p}{,}\PY{l+s+s2}{\PYZdq{}example\PYZdq{}}\PY{p}{)}\PY{p}{.}
\PY{n+nf}{condition\PYZus{}requirement}\PY{p}{(}\PY{l+s+s2}{\PYZdq{}x153\PYZdq{}}\PY{p}{,}\PY{l+s+s2}{\PYZdq{}variant\PYZdq{}}\PY{p}{,}\PY{l+s+s2}{\PYZdq{}example\PYZdq{}}\PY{p}{,}\PY{l+s+s2}{\PYZdq{}bzip\PYZdq{}}\PY{p}{,}\PY{l+s+s2}{\PYZdq{}True\PYZdq{}}\PY{p}{)}\PY{p}{.}
\PY{n+nf}{imposed\PYZus{}constraint}\PY{p}{(}\PY{l+s+s2}{\PYZdq{}x153\PYZdq{}}\PY{p}{,}\PY{l+s+s2}{\PYZdq{}depends\PYZus{}on\PYZdq{}}\PY{p}{,}\PY{l+s+s2}{\PYZdq{}example\PYZdq{}}\PY{p}{,}\PY{l+s+s2}{\PYZdq{}bzip2\PYZdq{}}\PY{p}{)}
\end{Verbatim}
\end{minipage}}\vspace{0.8em}
The \asp{condition\_requirement} follows a similar pattern to \asp{attr}, in
that it is a higher-order, variable arity relation to allow for uniform handling
of the various requirements.

In order to attach attributes to nodes conditionally (\ie{} when a
\texttt{when} spec is satisfied), the concretizer includes a set of rules for
conditionally imposed constraints on nodes. These rules hinge on two relations,
\asp{impose} and \asp{imposed\_constraint}. The latter is a variable-arity
relation taking advantage of the higher order encoding of \asp{attr} indexed by
a unique identifier, while the former controls constraints associated with that
identifier that are imposed on a particular node. What follows is an example of how
these facts are used in the logic program for the concretizer:
\par\vspace{0.8em}\noindent\colorbox{whitesmoke}{\begin{minipage}{0.98\columnwidth}\begin{Verbatim}[commandchars=\\\{\},fontsize=\footnotesize]
\PY{n+nf}{attr}\PY{p}{(}\PY{n+nv}{Attr}\PY{p}{,} \PY{n+nf}{node}\PY{p}{(}\PY{n+nv}{Name}\PY{p}{)}\PY{p}{,} \PY{n+nv}{Arg1}\PY{p}{)} \PY{p}{:\PYZhy{}}
  \PY{n+nf}{impose}\PY{p}{(}\PY{n+nv}{ID}\PY{p}{,} \PY{n+nf}{node}\PY{p}{(}\PY{n+nv}{Name}\PY{p}{)}\PY{p}{)}\PY{p}{,}
  \PY{n+nf}{imposed\PYZus{}constraint}\PY{p}{(}\PY{n+nv}{ID}\PY{p}{,} \PY{n+nv}{Name}\PY{p}{,} \PY{n+nv}{Arg}\PY{p}{)}\PY{p}{,} \PY{n+nf}{node}\PY{p}{(}\PY{n+nv}{Name}\PY{p}{)}\PY{p}{.}
\PY{n+nf}{attr}\PY{p}{(}\PY{n+nv}{Attr}\PY{p}{,} \PY{n+nf}{node}\PY{p}{(}\PY{n+nv}{Name}\PY{p}{)}\PY{p}{,} \PY{n+nv}{Arg1}\PY{p}{,} \PY{n+nv}{Arg2}\PY{p}{)} \PY{p}{:\PYZhy{}}
  \PY{n+nf}{impose}\PY{p}{(}\PY{n+nv}{ID}\PY{p}{,} \PY{n+nf}{node}\PY{p}{(}\PY{n+nv}{Name}\PY{p}{)}\PY{p}{)}\PY{p}{,}
  \PY{n+nf}{imposed\PYZus{}constraint}\PY{p}{(}\PY{n+nv}{ID}\PY{p}{,} \PY{n+nv}{Name}\PY{p}{,} \PY{n+nv}{Arg1}\PY{p}{,} \PY{n+nv}{Arg2}\PY{p}{)}\PY{p}{,} \PY{n+nf}{node}\PY{p}{(}\PY{n+nv}{Name}\PY{p}{)}\PY{p}{.}
\end{Verbatim}
\end{minipage}}\vspace{0.8em}
Note that there are similar rules at every potential arity of the \asp{attr}
relation. \asp{imposed\_constraint} is used to encode various package facts,
especially for reusable specs and \asp{impose} is derived when the conditions
associated with the \asp{ID} are present. In particular to this work, the
\asp{ID} considered will be that of the hash of a reusable spec.

\subsubsection{Reusing concrete specs}
\label{sec:concretizer-reuse}
In order to reuse concrete specs, the concretizer must be aware of installed specs
indexed by their unique hash) and which constraints would
be imposed in their reuse. This is encoded directly by providing an
\asp{installed\_hash} relation that ties package names to the hashes of
concrete specs and \asp{imposed\_constraint} facts for all of the attributes
of the concrete spec. What follows is the encoding of the concrete spec from
\cref{sec:spack-concretizer}:
\par\vspace{0.8em}\noindent\colorbox{whitesmoke}{\begin{minipage}{0.98\columnwidth}\begin{Verbatim}[commandchars=\\\{\},fontsize=\footnotesize]
\PY{n+nf}{installed\PYZus{}hash}\PY{p}{(}\PY{l+s+s2}{\PYZdq{}example\PYZdq{}}\PY{p}{,} \PY{l+s+s2}{\PYZdq{}abcdef1234\PYZdq{}}\PY{p}{)}\PY{p}{.}
\PY{n+nf}{imposed\PYZus{}constraint}\PY{p}{(}\PY{l+s+s2}{\PYZdq{}abcd1234\PYZdq{}}\PY{p}{,}\PY{l+s+s2}{\PYZdq{}version\PYZdq{}}\PY{p}{,}\PY{l+s+s2}{\PYZdq{}example\PYZdq{}}\PY{p}{,}\PY{l+s+s2}{\PYZdq{}1.1.0\PYZdq{}}\PY{p}{)}\PY{p}{.}
\PY{n+nf}{imposed\PYZus{}constraint}\PY{p}{(}\PY{l+s+s2}{\PYZdq{}abcd1234\PYZdq{}}\PY{p}{,}\PY{l+s+s2}{\PYZdq{}variant\PYZdq{}}\PY{p}{,}\PY{l+s+s2}{\PYZdq{}example\PYZdq{}}\PY{p}{,}\PY{l+s+s2}{\PYZdq{}bzip\PYZdq{}}\PY{p}{,}\PY{l+s+s2}{\PYZdq{}True\PYZdq{}}\PY{p}{)}\PY{p}{.}
\PY{n+nf}{imposed\PYZus{}constraint}\PY{p}{(}\PY{l+s+s2}{\PYZdq{}abcd1234\PYZdq{}}\PY{p}{,}\PY{l+s+s2}{\PYZdq{}node\PYZus{}os\PYZdq{}}\PY{p}{,}\PY{l+s+s2}{\PYZdq{}example\PYZdq{}}\PY{p}{,}\PY{l+s+s2}{\PYZdq{}centos8\PYZdq{}}\PY{p}{)}\PY{p}{.}
\PY{n+nf}{imposed\PYZus{}constraint}\PY{p}{(}\PY{l+s+s2}{\PYZdq{}abcd1234\PYZdq{}}\PY{p}{,}\PY{l+s+s2}{\PYZdq{}node\PYZus{}target\PYZdq{}}\PY{p}{,}\PY{l+s+s2}{\PYZdq{}example\PYZdq{}}\PY{p}{,}\PY{l+s+s2}{\PYZdq{}skylake\PYZdq{}}\PY{p}{)}\PY{p}{.}
\PY{n+nf}{imposed\PYZus{}constraint}\PY{p}{(}\PY{l+s+s2}{\PYZdq{}abcd1234\PYZdq{}}\PY{p}{,}\PY{l+s+s2}{\PYZdq{}depends\PYZus{}on\PYZdq{}}\PY{p}{,}\PY{l+s+s2}{\PYZdq{}example\PYZdq{}}\PY{p}{,}\PY{l+s+s2}{\PYZdq{}bzip2\PYZdq{}}\PY{p}{)}\PY{p}{.}
\PY{n+nf}{imposed\PYZus{}constraint}\PY{p}{(}\PY{l+s+s2}{\PYZdq{}abcd1234\PYZdq{}}\PY{p}{,}\PY{l+s+s2}{\PYZdq{}hash\PYZdq{}}\PY{p}{,}\PY{l+s+s2}{\PYZdq{}bzip2\PYZdq{}}\PY{p}{,}\PY{l+s+s2}{\PYZdq{}1234abcd\PYZdq{}}\PY{p}{)}\PY{p}{.}
\end{Verbatim}
\end{minipage}}\vspace{0.8em}
We will update this encoding in \cref{sec:impl-new-encoding} with some
indirection to better facilitate splicing.

In order to select reusable specs, the concretizer may select at most one spec
for every node with a potential candidate (\ie{} the \asp{installed\_hash} fact
is present for a node with that name). If chosen, the concretizer imposes all
constraints associated with that concrete spec. Any node which is not selected
for reuse is then marked to be built instead. Then, the highest optimization
objective in Spack's concretizer is to reuse as many specs as possible, or
equivalently to minimize the number of specs needing to be built:
\par\vspace{0.8em}\noindent\colorbox{whitesmoke}{\begin{minipage}{0.98\columnwidth}\begin{Verbatim}[commandchars=\\\{\},fontsize=\footnotesize]
\PY{p}{\PYZob{}} \PY{n+nf}{attr}\PY{p}{(}\PY{l+s+s2}{\PYZdq{}hash\PYZdq{}}\PY{p}{,} \PY{n+nf}{node}\PY{p}{(}\PY{n+nv}{Name}\PY{p}{)}\PY{p}{,} \PY{n+nv}{Hash}\PY{p}{)}\PY{l+s+sAtom}{:} \PY{n+nf}{installed\PYZus{}hash}\PY{p}{(}\PY{n+nv}{Name}\PY{p}{,} \PY{n+nv}{Hash}\PY{p}{)} \PY{p}{\PYZcb{}} \PY{l+m+mi}{1}
  \PY{p}{:\PYZhy{}} \PY{n+nf}{node}\PY{p}{(}\PY{n+nv}{Name}\PY{p}{)}\PY{p}{.}
\PY{n+nf}{impose}\PY{p}{(}\PY{n+nv}{Hash}\PY{p}{,} \PY{n+nv}{Node}\PY{p}{)} \PY{p}{:\PYZhy{}} \PY{n+nf}{attr}\PY{p}{(}\PY{l+s+s2}{\PYZdq{}hash\PYZdq{}}\PY{p}{,} \PY{n+nv}{Node}\PY{p}{,} \PY{n+nv}{Hash}\PY{p}{)}\PY{p}{,} \PY{n+nv}{Node}\PY{p}{.}
\PY{n+nf}{build}\PY{p}{(}\PY{n+nv}{Node}\PY{p}{)} \PY{p}{:\PYZhy{}} \PY{n+nv}{Node}\PY{p}{,} \PY{o}{not} \PY{n+nf}{attr}\PY{p}{(}\PY{l+s+s2}{\PYZdq{}hash\PYZdq{}}\PY{p}{,} \PY{n+nv}{Node}\PY{p}{,} \PY{n+nv}{Hash}\PY{p}{)}\PY{p}{.}
\PY{l+s+sAtom}{\PYZsh{}}\PY{l+s+sAtom}{minimize} \PY{p}{\PYZob{}} \PY{l+m+mi}{100}\PY{p}{,} \PY{n+nv}{PackageNode} \PY{l+s+sAtom}{:} \PY{n+nf}{build}\PY{p}{(}\PY{n+nv}{PackageNode}\PY{p}{)} \PY{p}{\PYZcb{}}
\end{Verbatim}
\end{minipage}}\vspace{0.8em}
\subsection{Augmenting Spack's Package Language}
\label{sec:impl-new-directive}
\subsubsection{Representation in the packaging DSL}
We add a single directive to Spack's packaging language for specifying
ABI-compatibility relationships between packages, \directive{can\_splice}.  The
bottom of \cref{fig:example-pkg} demonstrates \directive{can\_splice} directives
for our example package. The first spec argument describes the constraints on
the target of the splice (\ie{} the spec that would be replaced by the splice) and
the \texttt{when} spec argument describes the constraints on the package for the
splice to be valid. Both arguments support the full spec syntax. Note that
splices can be specified between two versions of the same package as well as
different packages. \directive{can\_splice} is a natural extension to the
packaging language since it utilizes the same API as most other Spack
directives.

This directive inverts the dependency structure of Spack when
considering binary distributions of packages. While the \directive{depends\_on}
relationship is given by the \emph{depending} package, \directive{can\_splice} allows
\emph{dependent} packages to state when it would not be ABI-breaking for them to
be swapped for another package. One subtlety to note is that packages state which
specs they \emph{can replace} rather than by which they \emph{can be replaced}.
This is an acknowledgment of the fact that developers of a common ABI, like
that of \package{MPICH}, will not necessarily be aware of every ABI-compatible
replacement, but developers of these replacements will know that their package
is ABI compatible.

\subsubsection{Encoding in ASP}
Unlike other directives in Spack, rather than being encoded as facts to describe
the constraints imposed by the \directive{can\_splice} directive, we choose to
encode the constraints from \directive{can\_splice} as a specialized rule.
\Cref{fig:can-splice-encoding} demonstrates our ASP encoding for the second
\directive{can\_splice} directive in \cref{fig:example-pkg}. It is a single rule
that derives the fact \asp{can\_splice}. This fact represents that the package
node in its first argument can replace a reusable spec with the name of the
second argument and the hash of the third argument, and can only be derived if
there is a package node satisfying the \texttt{when} spec and a pre-built spec
satisfying the target. It is a near-direct translation of the
\directive{can\_splice} directive to ASP. Being able to match node attributes to
pre-built spec attributes is one of factors that motivated the change in
encoding of reusable specs to use \asp{hash\_attr}.

\subsection{Changing the Encoding of Reusable Specs}
\label{sec:impl-new-encoding}
Recall from \cref{sec:concretizer-reuse} that reusable specs were encoded by
directly as \asp{imposed\_constraint} facts for every attribute of the concrete
spec, indexed by its hash. Since splicing changes the dependencies of a concrete
spec, the concretizer needs to be able to change which dependencies are imposed
by the spliced spec. Yet, in the previous encoding, all constraints from the
original spec are imposed immediately, and ASP does not have a mechanism for
removing derived facts from a model. Therefore, we changed the encoding of
pre-installed specs to add a layer of indirection between the encoding of the
attributes of reusable specs, and when those attributes are actually imposed.
%
\begin{figure}
  \begin{subfigure}[t]{1.0\columnwidth}
\input{code8.tex}
\caption{The new encoding of reusable specs.}
    \label{fig:new-encoding-hash-attrs}
  \end{subfigure}
  \begin{subfigure}[b]{1.0\columnwidth}
\begin{Verbatim}[commandchars=\\\{\},fontsize=\footnotesize]
\PY{n+nf}{imposed\PYZus{}constraint}\PY{p}{(}\PY{n+nv}{Hash}\PY{p}{,} \PY{n+nv}{Attr}\PY{p}{,} \PY{n+nv}{Name}\PY{p}{)} \PY{p}{:\PYZhy{}} \PY{n+nf}{hash\PYZus{}attr}\PY{p}{(}\PY{n+nv}{Hash}\PY{p}{,} \PY{n+nv}{Attr}\PY{p}{,} \PY{n+nv}{Name}\PY{p}{)}\PY{p}{.}
\PY{n+nf}{imposed\PYZus{}constraint}\PY{p}{(}\PY{n+nv}{Hash}\PY{p}{,} \PY{n+nv}{Attr}\PY{p}{,} \PY{n+nv}{Name}\PY{p}{,} \PY{n+nv}{Arg1}\PY{p}{)} \PY{p}{:\PYZhy{}}
  \PY{n+nf}{hash\PYZus{}attr}\PY{p}{(}\PY{n+nv}{Hash}\PY{p}{,} \PY{n+nv}{Attr}\PY{p}{,} \PY{n+nv}{Name}\PY{p}{,} \PY{n+nv}{Arg1}\PY{p}{)}\PY{p}{,}
  \PY{n+nv}{Arg1} \PY{p}{!}\PY{o}{=} \PY{l+s+s2}{\PYZdq{}hash\PYZdq{}}\PY{p}{.}
\PY{n+nf}{imposed\PYZus{}constraint}\PY{p}{(}\PY{n+nv}{Hash}\PY{p}{,} \PY{n+nv}{Attr}\PY{p}{,} \PY{n+nv}{PackageName}\PY{p}{,} \PY{n+nv}{Arg1}\PY{p}{,} \PY{n+nv}{Arg2}\PY{p}{)} \PY{p}{:\PYZhy{}}
  \PY{n+nf}{hash\PYZus{}attr}\PY{p}{(}\PY{n+nv}{Hash}\PY{p}{,} \PY{n+nv}{Attr}\PY{p}{,} \PY{n+nv}{PackageName}\PY{p}{,} \PY{n+nv}{Arg1}\PY{p}{,} \PY{n+nv}{Arg2}\PY{p}{)}\PY{p}{,}
  \PY{n+nv}{Attr} \PY{p}{!}\PY{o}{=} \PY{l+s+s2}{\PYZdq{}depends\PYZus{}on\PYZdq{}}\PY{p}{.}
\PY{n+nf}{imposed\PYZus{}constraint}\PY{p}{(}\PY{n+nv}{ParentHash}\PY{p}{,}\PY{l+s+s2}{\PYZdq{}hash\PYZdq{}}\PY{p}{,}\PY{n+nv}{ChildName}\PY{p}{,}\PY{n+nv}{ChildHash}\PY{p}{)} \PY{p}{:\PYZhy{}}
  \PY{n+nf}{hash\PYZus{}attr}\PY{p}{(}\PY{n+nv}{ParentHash}\PY{p}{,}\PY{l+s+s2}{\PYZdq{}hash\PYZdq{}}\PY{p}{,}\PY{n+nv}{ChildName}\PY{p}{,}\PY{n+nv}{ChildHash}\PY{p}{)}\PY{p}{,}
  \PY{n+nv}{ChildHash}\PY{p}{!}\PY{o}{=}\PY{n+nv}{ParentHash}\PY{p}{,}
  \PY{o}{not} \PY{n+nf}{can\PYZus{}splice}\PY{p}{(}\PY{n+nf}{node}\PY{p}{(}\PY{n+nv}{Name}\PY{p}{)}\PY{p}{,}\PY{n+nv}{ChildName}\PY{p}{,}\PY{n+nv}{ChildHash}\PY{p}{)}\PY{p}{.}
\PY{n+nf}{imposed\PYZus{}constraint}\PY{p}{(}\PY{n+nv}{ParentHash}\PY{p}{,}\PY{l+s+s2}{\PYZdq{}depends\PYZus{}on\PYZdq{}}\PY{p}{,}\PY{n+nv}{ParentName}\PY{p}{,}\PY{n+nv}{ChildName}\PY{p}{)} \PY{p}{:\PYZhy{}}
  \PY{n+nf}{hash\PYZus{}attr}\PY{p}{(}\PY{n+nv}{ParentHash}\PY{p}{,}\PY{l+s+s2}{\PYZdq{}depends\PYZus{}on\PYZdq{}}\PY{p}{,}\PY{n+nv}{ParentName}\PY{p}{,}\PY{n+nv}{ChildName}\PY{p}{)}\PY{p}{,}
  \PY{o}{not} \PY{n+nf}{can\PYZus{}splice}\PY{p}{(}\PY{n+nf}{node}\PY{p}{(}\PY{n+nv}{Name}\PY{p}{)}\PY{p}{,}\PY{n+nv}{ChildName}\PY{p}{,}\PY{n+nv}{ChildHash}\PY{p}{)}\PY{p}{.}
\end{Verbatim}

\caption{Rules that recover the \asp{imposed\_constraint} facts from the original
encoding of reusable specs.}
    \label{fig:new-encoding-imposed-constraints}
  \end{subfigure}
  \caption{The new encoding of reusable specs introduces indirection into the of \asp{imposed\_constraint}, this indirection is key for splicing to allow for
  splicing to change the dependencies of reusable specs.}
\label{fig:new-encoding}
\Description{ASP code demonstrating the change in encoding to already installed specs.}
\end{figure}

%
\Cref{fig:new-encoding-hash-attrs} demonstrates how we introduce this
indirection through \asp{hash\_attr} facts and
\cref{fig:new-encoding-imposed-constraints} shows how we recover the semantics
from the previous encoding. Note that the initial imposition of \asp{"hash"} and
\asp{"depends\_on"} depends on the presence of
\asp{can\_splice}. These attributes are where splices can be
introduced and this encoding allows the concretizer to make decisions about
imposition and reuse of either the original dependency or a spliced dependency.
Without this \asp{can\_splice}, there are no valid splices and
thus the concretizer defaults to its previous behavior of always reusing the
dependencies of any pre-built spec whenever it reuses the spec itself.
%
\subsection{Splicing in the Concretizer}
\label{sec:impl-splice-concretizer}
\begin{figure}
\begin{subfigure}[t]{1.0\columnwidth}
\begin{Verbatim}[commandchars=\\\{\},fontsize=\footnotesize]
\PY{c+c1}{\PYZpc{} can\PYZus{}splice(\PYZdq{}example\PYZhy{}ng@2.3.2+compat\PYZdq{},when=\PYZdq{}@1.1.0+compat)}
\PY{n+nf}{can\PYZus{}splice}\PY{p}{(}\PY{n+nf}{node}\PY{p}{(}\PY{l+s+s2}{\PYZdq{}example\PYZdq{}}\PY{p}{)}\PY{p}{,}\PY{l+s+s2}{\PYZdq{}example\PYZhy{}ng\PYZdq{}}\PY{p}{,}\PY{n+nv}{Hash}\PY{p}{)} \PY{p}{:\PYZhy{}}
  \PY{n+nf}{installed\PYZus{}hash}\PY{p}{(}\PY{l+s+s2}{\PYZdq{}example\PYZhy{}ng\PYZdq{}}\PY{p}{,}\PY{n+nv}{Hash}\PY{p}{)}\PY{p}{,}
  \PY{n+nf}{attr}\PY{p}{(}\PY{l+s+s2}{\PYZdq{}node\PYZdq{}}\PY{p}{,}\PY{n+nf}{node}\PY{p}{(}\PY{l+s+s2}{\PYZdq{}example\PYZdq{}}\PY{p}{)}\PY{p}{)}\PY{p}{,}
  \PY{n+nf}{hash\PYZus{}attr}\PY{p}{(}\PY{n+nv}{Hash}\PY{p}{,}\PY{l+s+s2}{\PYZdq{}version\PYZdq{}}\PY{p}{,}\PY{l+s+s2}{\PYZdq{}example\PYZhy{}ng\PYZdq{}}\PY{p}{,}\PY{l+s+s2}{\PYZdq{}2.3.2\PYZdq{}}\PY{p}{)}\PY{p}{,}
  \PY{n+nf}{attr}\PY{p}{(}\PY{l+s+s2}{\PYZdq{}version\PYZdq{}}\PY{p}{,}\PY{n+nf}{node}\PY{p}{(}\PY{l+s+s2}{\PYZdq{}example\PYZdq{}}\PY{p}{)}\PY{p}{,}\PY{l+s+s2}{\PYZdq{}1.1.0\PYZdq{}}\PY{p}{)}\PY{p}{,}
  \PY{n+nf}{hash\PYZus{}attr}\PY{p}{(}\PY{n+nv}{Hash}\PY{p}{,}\PY{l+s+s2}{\PYZdq{}variant\PYZdq{}}\PY{p}{,}\PY{l+s+s2}{\PYZdq{}example\PYZhy{}ng\PYZdq{}}\PY{p}{,}\PY{l+s+s2}{\PYZdq{}compat\PYZdq{}}\PY{p}{,}\PY{l+s+s2}{\PYZdq{}True\PYZdq{}}\PY{p}{)}\PY{p}{,}
  \PY{n+nf}{attr}\PY{p}{(}\PY{l+s+s2}{\PYZdq{}variant\PYZdq{}}\PY{p}{,}\PY{n+nf}{node}\PY{p}{(}\PY{l+s+s2}{\PYZdq{}example\PYZdq{}}\PY{p}{)}\PY{p}{,}\PY{l+s+s2}{\PYZdq{}compat\PYZdq{}}\PY{p}{,}\PY{l+s+s2}{\PYZdq{}True\PYZdq{}}\PY{p}{)}\PY{p}{.}
\end{Verbatim}

\caption{Compiled ASP encoding of the \directive{can\_splice} directive included
  as a comment on the first line of the fragment}
\label{fig:can-splice-encoding}
\end{subfigure}
\begin{subfigure}[b]{1.0\columnwidth}
\begin{Verbatim}[commandchars=\\\{\},numbers=left,xleftmargin=5mm,fontsize=\footnotesize]
\PY{p}{\PYZob{}} \PY{n+nf}{imposed\PYZus{}constraint}\PY{p}{(}\PY{n+nv}{ParentHash}\PY{p}{,}\PY{l+s+s2}{\PYZdq{}hash\PYZdq{}}\PY{p}{,}\PY{n+nv}{DepName}\PY{p}{,}\PY{n+nv}{DepHash}\PY{p}{)} \PY{p}{\PYZcb{}} \PY{p}{:\PYZhy{}} 
  \PY{n+nf}{hash\PYZus{}attr}\PY{p}{(}\PY{n+nv}{ParentHash}\PY{p}{,}\PY{l+s+s2}{\PYZdq{}hash\PYZdq{}}\PY{p}{,}\PY{n+nv}{DepName}\PY{p}{,}\PY{n+nv}{DepHash}\PY{p}{)}\PY{p}{,}
  \PY{n+nf}{can\PYZus{}splice}\PY{p}{(}\PY{n+nf}{node}\PY{p}{(}\PY{n+nv}{SpliceName}\PY{p}{)}\PY{p}{,}\PY{n+nv}{DepName}\PY{p}{,}\PY{n+nv}{DepHash}\PY{p}{)}\PY{p}{.}
\PY{n+nf}{attr}\PY{p}{(}\PY{l+s+s2}{\PYZdq{}splice\PYZdq{}}\PY{p}{,}\PY{n+nv}{ParentNode}\PY{p}{,}\PY{n+nf}{node}\PY{p}{(}\PY{n+nv}{SpliceName}\PY{p}{)}\PY{p}{,}\PY{n+nv}{DepName}\PY{p}{,}\PY{n+nv}{DepHash}\PY{p}{)} \PY{p}{:\PYZhy{}} 
  \PY{n+nf}{attr}\PY{p}{(}\PY{l+s+s2}{\PYZdq{}hash\PYZdq{}}\PY{p}{,}\PY{n+nv}{ParentNode}\PY{p}{,}\PY{n+nv}{ParentHash}\PY{p}{)}\PY{p}{,}
  \PY{n+nf}{hash\PYZus{}attr}\PY{p}{(}\PY{n+nv}{ParentHash}\PY{p}{,}\PY{l+s+s2}{\PYZdq{}hash\PYZdq{}}\PY{p}{,}\PY{n+nv}{DepName}\PY{p}{,}\PY{n+nv}{DepHash}\PY{p}{)}\PY{p}{,}
  \PY{n+nf}{can\PYZus{}splice}\PY{p}{(}\PY{n+nf}{node}\PY{p}{(}\PY{n+nv}{SpliceName}\PY{p}{)}\PY{p}{,}\PY{n+nv}{DepName}\PY{p}{,}\PY{n+nv}{DepHash}\PY{p}{)}\PY{p}{,}
  \PY{o}{not} \PY{n+nf}{imposed\PYZus{}constraint}\PY{p}{(}\PY{n+nv}{ParentHash}\PY{p}{,}\PY{l+s+s2}{\PYZdq{}hash\PYZdq{}}\PY{p}{,}\PY{n+nv}{DepName}\PY{p}{,}\PY{n+nv}{DepHash}\PY{p}{)}\PY{p}{.}
\PY{n+nf}{imposed\PYZus{}constraint}\PY{p}{(}\PY{n+nv}{Hash}\PY{p}{,}\PY{l+s+s2}{\PYZdq{}depends\PYZus{}on\PYZdq{}}\PY{p}{,}\PY{n+nv}{ParentName}\PY{p}{,}\PY{n+nv}{SpliceName}\PY{p}{,}\PY{n+nv}{Type}\PY{p}{)} \PY{p}{:\PYZhy{}} 
  \PY{n+nf}{hash\PYZus{}attr}\PY{p}{(}\PY{n+nv}{Hash}\PY{p}{,}\PY{l+s+s2}{\PYZdq{}depends\PYZus{}on\PYZdq{}}\PY{p}{,}\PY{n+nv}{ParentName}\PY{p}{,}\PY{n+nv}{DepName}\PY{p}{,}\PY{n+nv}{Type}\PY{p}{)}\PY{p}{,}
  \PY{n+nf}{hash\PYZus{}attr}\PY{p}{(}\PY{n+nv}{Hash}\PY{p}{,}\PY{l+s+s2}{\PYZdq{}hash\PYZdq{}}\PY{p}{,}\PY{n+nv}{DepName}\PY{p}{,}\PY{n+nv}{DepHash}\PY{p}{)}\PY{p}{,}
  \PY{n+nf}{attr}\PY{p}{(}\PY{l+s+s2}{\PYZdq{}splice\PYZdq{}}\PY{p}{,}\PY{n+nf}{node}\PY{p}{(}\PY{n+nv}{ParentName}\PY{p}{)}\PY{p}{,}\PY{n+nf}{node}\PY{p}{(}\PY{n+nv}{SpliceName}\PY{p}{)}\PY{p}{,}\PY{n+nv}{DepName}\PY{p}{,}\PY{n+nv}{DepHash}\PY{p}{)}\PY{p}{.}
\end{Verbatim}

\caption{The core logic implementing splicing. This fragment selects whether to
  execute a splice based on the presence of the \asp{can\_splice} fact, and then
  imposes the new spliced dependency instead of the pre-installed spec's original
  dependency.} 
  \label{fig:splice-lp}
\end{subfigure}
\caption{The splicing implementation in ASP portion of Spack's concretizer.}
\label{fig:splice-concretizer}
\Description{An ASP snippet implementing the core of the splicing logic.}
\end{figure}



%
\Cref{fig:splice-lp} demonstrates the core fragment of the concretizer
implementing splicing. The rules can be read top to bottom as implementing the
fairly simple decision procedure for splicing. The rules beginning on lines 1
and 4 describe an exclusive-or condition for how to impose the dependencies of the spec
that is being chosen for reuse provided it has a potential candidate for a
splice. When reusing a concrete spec, the concretizer may impose its original
dependency (the choice on line 1). If it does not (the condition on line 8), it
must instead splice in a compatible dependency (the conclusion on line 4). If it
decides to splice in a new dependency (the condition on line 12), that new
dependency is imposed instead of the original one (the conclusion on line 9).

Perhaps it is surprising how little logic is required to implement the core
splicing in the concretizer's logic program. We owe this elegance to the change
in encoding of reusable concrete specs. The small indirection added through
\asp{hash\_attr} facts provides the perfect hook for changing the dependency
when splicing. By refactoring machinery already present in the concretizer (\ie{}
\asp{impose} and \asp{imposed\_constraint}), we are still able to take advantage
of the logic that maintains the integrity of the whole spec-DAG (\eg{} ensuring
compatible microarchitectures among all specs). After making this encoding
change, Spack core developers have found subsequent uses in the concretizer for
indirection in imposing concrete specs\footnote{https://github.com/spack/spack/pull/45189}.

Ultimately, the concretizer produces a mapping between pre-built specs (\ie{}
nodes with the \asp{"hash"} attribute) and their dependencies that will be
replaced by splicing (\ie{} nodes with the \asp{"splice"} attribute). This
mapping can be used directly not only to build the spliced specs in a
straightforward manner but also in the rewiring of spliced specs, as
described in \cref{sec:splice-relocation}. Finally, the splicing logic can be
conditionally loaded, and thus is a fully transparent opt-in feature. We
validate that our encoding change for pre-installed specs incurs minimal
overhead in the next section.

